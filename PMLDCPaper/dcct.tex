\documentclass[preprint]{sigplanconf}

% The following \documentclass options may be useful:

% preprint      Remove this option only once the paper is in final form.
% 10pt          To set in 10-point type instead of 9-point.
% 11pt          To set in 11-point type instead of 9-point.
% numbers       To obtain numeric citation style instead of author/year.

\usepackage{amsmath}

\newcommand{\cL}{{\cal L}}

\begin{document}

\special{papersize=8.5in,11in}
\setlength{\pdfpageheight}{\paperheight}
\setlength{\pdfpagewidth}{\paperwidth}

\conferenceinfo{CONF 'yy}{Month d--d, 20yy, City, ST, Country}
\copyrightyear{20yy}
\copyrightdata{978-1-nnnn-nnnn-n/yy/mm}
\copyrightdoi{nnnnnnn.nnnnnnn}

% Uncomment the publication rights you want to use.
%\publicationrights{transferred}
%\publicationrights{licensed}     % this is the default
%\publicationrights{author-pays}

%\titlebanner{banner above paper title}        % These are ignored unless
\preprintfooter{Ensuring concurrent operations on critical data preserve its
  invariants and expected semantics for all concurrent clients}   % 'preprint' option specified.

\title{Data-centric Consistency Policies}
\subtitle{A new theory and programming model for developing
  distributed applications}

\authorinfo{Nosheen Zaza}
           {Università della Svizzera italiana (USI)}
           {zazan@usi.ch}
\authorinfo{Nate Nystrom}
           {Università della Svizzera italiana (USI)}
           {nystrom@usi.ch}

\maketitle

\begin{abstract}
The consistency level enforced by operations on replicated, distributed data is 
an important parameter in distributed applications, as it impacts correctness, 
performance and availability. Nowadays, it is very common to find various consistency
levels used in a single application, however, current frameworks do not facilitate
working with and reasoning about different consistency properties of an application. 
Furthermore, the problem of defining the semantics of operations enforcing different 
consistency levels has not been well defined or studied. We propose a new
approach for specifying consistency properties based on an important observation: 
correctness criteria and invariants are a property of data, not operations, hence it is 
reasonable to enable defining the consistency properties required to enforce them on data 
rather than operations. We have defined a theoretical model for data-centric reasoning about consistency
properties, and designed a domain-specific programming language we call Data Centric Cloud Types, that enables 
declaring consistency properties, along with a type system to check the compatibility of declared consistency 
properties. The benefits of this language are programs with clearer consistency properties which are 
easier to define and reason about, and the possibility of optimising runtime and commit protocols to enable 
exploiting consistency property definitions to enhance performance.
\end{abstract}

\category{CR-number}{subcategory}{third-level}

% general terms are not compulsory anymore,
% you may leave them out
%\terms
%term1, term2

\keywords
distributed systems, domain-specific programming langauges

\section{Introduction}

The text of the paper begins here.

\section{System Model}
\appendix
\section{Appendix Title}

This is the text of the appendix, if you need one.

\acks

Acknowledgments, if needed.

% We recommend abbrvnat bibliography style.

\bibliographystyle{abbrvnat}

% The bibliography should be embedded for final submission.

\begin{thebibliography}{}
\softraggedright

\bibitem[Smith et~al.(2009)Smith, Jones]{smith02}
P. Q. Smith, and X. Y. Jones. ...reference text...

\end{thebibliography}


\end{document}
