\documentclass[]{usiinfprospectus}
\usepackage[all]{xy}
\usepackage{amsthm}
% \usepackage{amssymb}
\usepackage{listings}
\usepackage{color}
\usepackage{multicol}

%\usepackage{texnansi}
%\usepackage{gtamacgeorgia}


%\fontencoding{T1}
%\fontfamily{garamond}
%\fontseries{m}
%\fontshape{it}
%\fontsize{12}{15}
%\selectfont

%\renewcommand{\encodingdefault}{T1}
%\renewcommand{\rmdefault}{garamond}

%\addtolength{\itemsep}{-\baselineskip}

\lstdefinelanguage{imp}{
  morekeywords={%
          skip,pass,if,then,else,do,while,and,or,not,true,false},
  otherkeywords={},
  sensitive=true,
  morecomment=[l]{//},
  morecomment=[n]{/*}{*/},
  morestring=[b]",
  morestring=[b]',
  morestring=[b]"""
}[keywords,comments,strings]
 


\lstdefinelanguage{scala}{
  morekeywords={%
          abstract,case,catch,class,def,do,else,extends,%
          false,final,finally,for,forSome,if,implicit,import,lazy,%
          match,new,null,object,override,package,private,protected,%
          return,macro, sealed,super,this,throw,trait,true,try,type,%
          val,var,while,with,yield},
  otherkeywords={=>,<-,<\%,<:,>:,\#,@},
  sensitive=true,
  morecomment=[l]{//},
  morecomment=[n]{/*}{*/},
  morestring=[b]",
  morestring=[b]',
  morestring=[b]"""
}[keywords,comments,strings]
 

\definecolor{dkgreen}{rgb}{0,0.6,0}
\definecolor{gray}{rgb}{0.5,0.5,0.5}
\definecolor{mauve}{rgb}{0.58,0,0.82}
 
% Default settings for code listings
\lstset{frame=tb,
  language=scala,
  aboveskip=3mm,
  belowskip=3mm,
  showstringspaces=false,
  columns=flexible,
  numbers=none,
  keywordstyle=\color{blue},
  commentstyle=\color{dkgreen},
  stringstyle=\color{mauve},
  frame=none,
  breaklines=true,
  breakatwhitespace=true
  tabsize=2
}
% \lstdefinestyle{basic}{showstringspaces=false,columns=fullflexible,language=haskell,xleftmargin=1pc,%
% %basicstyle=\small\bfseries\itshape,%
% %keywordstyle=\underbar,
% mathescape=true,%
% lineskip=-1pt,%
% keepspaces=true,%
% %basicstyle=\small\sffamily,%
% basicstyle=\small\ttfamily,%
% commentstyle=\mdseries\rmfamily,%
% %moredelim=**[is][\color{white}]{~}{~}
% }
% \lstset{style=basic}


\newcommand\stepsone{\longrightarrow}
\newcommand\bnf{\,\,|\,\,}

\newcommand\later[1]{{\color{dkgreen} {#1}}}
\newcommand\improve[1]{{\color{gray} {#1}}}
\newcommand\remove[1]{{\color{mauve} {#1}}}
\newcommand\comment[1]{{\color{red} {#1}}}



% \usepackage[T1]{fontenc}
% \usepackage{lmodern}
% \usepackage{lmodern}
\captionsetup{labelfont={bf}}

\author{Nosheen Zaza}


\lstset{language=scala}
\lstset{mathescape=false}

\title{Composable Data Consistency Policies}
% \subtitle{The (optional) sub title}
\versiondate{\today}

\begin{committee}
%With more than 1 advisor an error is raised...: only 1 advisor is allowed!
\researchadvisor[Universit\`a della Svizzera Italiana, Switzerland]{Prof.}{Nate}{Nystrom}
\academicadvisor[Universit\`a della Svizzera Italiana, Switzerland]{Prof.}{Nate}{Nystrom}
\committeemember[Universit\`a della Svizzera Italiana, Switzerland]{Prof.}{Committee}{Member1}
\committeemember[Universit\`a della Svizzera Italiana, Switzerland]{Prof.}{Committee}{Member2}
%\committeemember[Universit\`a della Svizzera Italiana, Switzerland]{Prof.}{Committee}{Member3}
% \coadvisor[Universit\`a della Svizzera Italiana, Switzerland]{Prof.}{Research}{Co-Advisor1}
% \coadvisor[Universit\`a della Svizzera Italiana, Switzerland]{Prof.}{Research}{Co-Advisor2}
%\coadvisor[Universit\`a della Svizzera Italiana, Switzerland]{Prof.}{Research}{Co-Advisor3}
\phddirector{Prof.}{Stefan}{Wolf}
\end{committee}

%!TEX root = prospectus.tex

\abstract {
The importance of relaxed consistency models is becoming evident, with an increasing variety  of  systems trading off strong consistency for availability and performance. Developing such systems is a challenging, because consistency requirements may vary across system components or for different system clients. Even when uniform consistency is required, the underlying environment may provide weaker or stronger consistency than desired, which requires complex interaction patterns between the environment and the system to obtain the desired consistency properties . A consequential, and potentially harder challenge to address is that the interaction between entities enforcing diverse consistency policies must not violate any of them. Ongoing research has produced several methodologies, tools, and programming models that assist programmers to reason about, express and check consistency policies. In this research prospectus, we overview contemporary research on managing consistency, and motivate the need for further research we plan to conduct in this area. We plan to explore two fundamental gaps with current research , One is that consistency issues are studies in isolation for in large distributed systems, or for low level memory operations, this prevents developing a unified of theories and tools to cover all such issues, which produces duplication of efforts, and tools that are not elegant. The second problem is that there are no clear semantics for composing, also current tools data abstractions make it hard to attach desired policies.
}


\begin{document}
\maketitle


%!TEX root = prospectus.tex
\section{Introduction and Problem Domain}
\subsection {Replication and Partitioning}

Data replication and partitioning appear in hardware and software systems at every scale. Consider, on a small scale, the replication of data among caches and main memory, or partitioning program data among objects in objects oriented systems, to be later composed and used as logically meaningful modules. Replicating data across web servers, or partitioning data across tables in a database are some larger scale examples. 
Both replication and partitioning introduce several challenges to system designers and programmers. A major challenge is the problem of data consistency: that an update to a replica or partition will be propagated correctly to clients of other replicas or partitions. 

\subsection {Correctness of Concurrent Operations}
The notion of correctness means different things in different contexts, depending on different requirements. Traditionally (mostly before the internet era), correctness meant that systems behave as if all operations were performed serially at one replica or across all partitions, meaning that operations form a total order. examples of consistency models that possess this quality are one-copy serializability [Bernstein et al. 1987]. sequential consistency[Leslie some year] or linearizability[wing some year]. Enforcing a total order on operations occurring within a large distributed system, with many replicas can prove expensive, as well as unnecessary for many applications. In this era of massive, distributed online applications, Weaker models, such as eventual consistency [Sheth et al. 1991] or session consistency[Ref], are the norm not the exception. In fact, common issues in distributed systems, such as loss of data during communication or inavailability of processing nodes makes it challenging to enforce even those weaker models.  Weak consistency models can be found on a much smaller scale, in the absence of replication as well. Class ConcurrentSkipListMap[ref], a part of Java standard library, contains execution paths that do not correspond to any sequential execution, because some carefully characterized operations that lost races are not retired, the consistency of instances is enforced by the semantics of the class.

\subsection{Enforcing and Strengthening Consistency}

Enforcing data consistency requires coordinating update requests, and determining when those updates become visible to clients.  There are several mechanisms for achieving this, depending on the required consistency and performance properties such as hardware level CAS operations, explicit synchronization through locking, transactions… etc.  I want to say here that all these mechanisms must determine what to do on failure of operation, specify the retry policy




% \bibliographystyle{abbrv}
% \bibliography{../myBibliographyDB}



\end{document}
