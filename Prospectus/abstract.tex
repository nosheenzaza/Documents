%!TEX root = prospectus.tex

\abstract {
The importance of relaxed consistency models has become evident, with an increasing variety of  systems trading off strong consistency for availability and performance, such as eventually consistent replicated key-value stores, or algorithms exhibiting benign data races. Developing such systems is a challenging, because consistency requirements may vary across system components or for different system clients; even when uniform consistency is required, the underlying environment often provide weaker or stronger consistency than desired. Thus, developers have to encode complex interaction patterns among the environment and various system components to obtain the desired consistency properties for each component, without compromising consistency, performance or availability of other components. 

While ongoing research has produced several methodologies, tools, and programming models that assist developers reason about, express and check consistency policies, most work is polarized either around relaxed consistency models, as employed in distributed systems, or around strict models described in shared-memory concurrency literature. Hence, we do not have a unified theory to describe diverse, commonly used consistency policies on different scales, neither we have comprehensive semantics to describe how various consistency models interact or compose. In this research prospectus, we overview classic and recent work on managing consistency, and motivate the need for further research. We believe that studying consistency as a single, universal property of applications will pave the way to creating elegant, generic frameworks and programming languages to express a wide variety of consistency management patterns, in both distributed systems and multicore applications. We also believe that enabling programmers to accurately and precisely define consistency policies can lead  to better safety and performance properties.
}
