%! TEX root = main.tex


\section{The  Language} 

Our language is an extension to cloud types language, with the possibility of
declaring consistency sets. We will introduce the language through an extended
version of the example presented in the previous section. 

%I need to describe it through my somewhat complicated example, then show the
%possible orderings of operations based on the consistency levels. I will also
%show legal and illegal consistency policy nestings. Show possible mistakes that
%can be made when there are no policies. This can happen when I describe the
%implementation of the framework. The latter examples, should
%be done maybe after explaining the theory on consistency level combinations.
%Other than that. 
%
\begin{figure}[tp]
\begin{lstlisting}
entity Customer {
   name: CString 
   country: Country
   paymentMethod: CString
   paymentInfo: CString
}
@serial set (paymentInfo)
entity CustomerSettings {
   paymentMethod: CString references Customer.paymentMethod
   paymentInfo: CString references Customer.paymentInfo
}
@snapshot set payment (CustomerSettings.paymentMethod, CustomerSettings.paymentInfo)
array Product[id: string] {
  name: CString
  price: CInt
  @serial set quantity: CInt 
}
entity Order(customer: Customer) {
  time: CTime
}
array OrderItem[order: Order, product: Product] {
  quantity: CInt
  price: CInt
}
array CartItem[customer: Customer, product: Product] {
  quantity: CInt
}
entity OrderQueue {
  customer
  address // critical not to ship to an old address, hence serial, to force the
  latest version.
  amount
}
function addToCart(c: Customer, p: Product, q: int) {
  cartItem[c, p].quantity.add(q)
}
function deleteCustomer(c: Customer) {
  delete c
}
function submitOrder(customer: Customer) {
  var order = new Order(customer)
  order.time.set(now());

  foreach cartitem in entries CartItem.quantity 
    where cartitem.customer == customer {
      var oitem = OrderItem[order, cartitem.product]
      oitem.quantity = cartitem.quantity
      oitem.price = cartitem.quantity * cartitem.product.price
      cartitem.quantity.add(-oitem.quantity)
      product.quantity.add(-oitem.quantity))
  }
}
\end{lstlisting}
\caption{Customer database example with atomic set declarations}
\end{figure}
\label{lst:ctStoreSets}
\
%function showOrders(customer: Customer) {
%  foreach order in all Order
%    where order.Customer == customer
%    orderby order.time {    
%    Print("Order of" + order.time)
%    
%    foreach(i in all OrderItem) where i.order == order {
%       Print(i.quantity + " " 
%            + i.product.name + "for" + i.price)
%    }
%  }

\section{An Online Shopping Example}
A common design method for distributed data is to store data in a way that
minimizes the number of nodes to be contacted in order to reply to a
query. This often means denormalizing data. That is, columns may be replicated,
and materialized views are common. Bearing that in mind, let's add paymentMethod and paymentInfo 
properties to the Customer entity, and let's create a view, called
CustomerSettings, contains only the payment information. (a bit made up, but
representative enough for now). The fields of this entity are a view of the
corresponding fields in the Customers, they need not be the most recent version
of the original entity, however, they must be coherent. This is an example of 
multi-part data. Hence, we declare the source of this data, and also put it in a
set annotated with @snapshot. However, in the original table, these properties
are annotated @serial, this is because we must ensure that the items a customer
orders are paid or using the most recent payment method, however, for display,
we do not care much, in the worst case, the client will try to re-enter the
payment method (These are just example assumptions, the programmer is free to
use whatever policies they want). Attribute quantity in Product array is
annotated @serial, meaning all its replicas must be views consistently, and all
updates to it are serialized. This is in order to avoid selling the same item to
multiple clients, and to avoid distorting the stock value. All non annotated
fields are eventually consistent. \\

Let's examine the operations on our data definitions. addToCart is eventually
consistent, with no ACI guarantees. We assume it is ok if the cart state is not
consistent across all its replicas (e.g mobile and laptop of a customer), since
the final checkout will happen at one machine. Deleting a customer or adding it,
must be serial, since Customer entity contains a serial field. We explain the
reasons in more detail in the following section, but in short, if this was not
the case, the serializability of paymentInfo cannot be guaranteed. The most
interesting operation is submitOrder, which touches data in different sets with
different consistency policies. What should be the policy of the transaction
itself? In fact, the concept of a transaction in our language differs from that
in a regular DBMS. The only two restrictions we have on a transaction (a function
here), is that they must maintain the consistency of all data sets they touch,
and they signal success once the success of operations on all data sets is
acknowledges, each according to its own consistency policy. This gives a lot of
flexibility when committing or rolling back an operation. We guarantee that the
touched item quantity will be observed serially, both when committed or rolled
back, and we do not care for the rest. Note that the correctness of these
assumptions depends on other transactions running concurrently. 

The reader may have many questions by now. Is it possible to have transactions
at different consistency levels touch the same data sets? Is it possible to read
or write at consistency levels other than those declared? We answer all these
questions in the next section, when we describe our system model.

%This example introduces a data schema that is denormalized. Denormalization is a
%common strategy in industry when designing distributed data storage. This is
%because joining data from different tables, located on diverse nodes is a costly
%operation, hence every column (I need to explain somewhere what a column means
%in general) is duplicated. In fact, even in traditional relational systems,
%views of tables were supported, which is somehow similar to denormalizing data
%for performance (refs). In our language, we support declaring views along with
%their relationships and consistency policy in relation to the original tables.




