%! TEX root = main.tex


\section{Background and Influences}
This work build upon 3 pillars of work. The main idea to associate consistency
policies with data comes from the work on atomic sets, which is explained in
Section 2.1. The language we propose, that incorporates the notion of sets
annotated with consistency policies, is built on top of Cloud Types, explained
in Section 2.2. As an example of a consistency policy weaker than than
serializability but stronger than eventual consistency, we incorporated atomic
visibility into the language. This consistency level is explained in Section
2.3. Finally, the commit protocol is an extended version of MDCC(ref) protocol,
which is paxos-based, and attempts to commit transactions with linearizability
guarantees.
% Now that I think about it, I am not sure if this is the most suitable protocol
% for this framework. I find its use appealing because it does not require
% locking, and can exploit domain knowledge to optimize committing, however I
% will consult Fernando later to see if I can use a better protocol. Another
% thing that makes it appealing to me is that there is an available
% implementation in Java!

% Another way to talk about this chapter is to mention the ideas we built upon,
% and the research that motivated them, the ideas are: consistency as a property
% of data, consistency as a permanent property of data.

% Now I am thinking that there are two types of operations that happen to data,
% placement and manipulation, Thing is that the way a proposed updated value is
% calculated may have nothing to do with the object itself, here we do not care
% about that, but we do care about where it came from. 

\subsection{Atomic Sets}
At the heart of this work, is the concept of Atomic Sets (ref). Atomic sets allow 
expressing synchronization constraints in a
data-centric manner. Built as an extension to Java, they allow programmers to
declare sets of objects that require atomic (linearizable) access, in order to
maintain correctness invariants (e.g. each call to increment a counter is
applied). Fig.~\ref{lst:acCounter} shows an example of an atomic set
declaration. An atomic set, named \emph{a} is declared, and the field \emph{val}
is added to it. This indicates that all accesses to \emph{val} must be
serialized. Thus the compiler will insert synchronization blocks around all
public methods (referred to as units of work), serializing all accesses from
outside the object to that field. One could argue that it might be too strict to
serialize all accesses through public methods, this is not an unusual price to
pay in conservative (there is a better word here) type systems, but it is
necessary to guarantee type-safety. After compilation, this code will be
transformed as shown in Fig.~\ref{lst:acCounterCompiled}. \\

% TODO add a simplified version of the list example, or create an example that
% involves only two variables. It is good for myself though to look again at the
% list example later on.

% Looking at the example in the paper, it is somewhat complex compared to the
% crude description I have here, I may as well skip it altogether and just
% mention this simple explanation. 
The most remarkable aid this framework provides, is that after the progrmmer
It is possible to add
more than one field to an atomic set, and even add fields that belong to different
objects to the same set. It is also possible to create operation specific sets,
which is useful to avoid high-level data races. For example Fig.... shows the
code of method \emph{transfer}, which appends all elements in one list to
another list. To guarantee linearizability, one must ensure that the source and destination
sets do not change while this operation is progressing. The compiler inserts
code to acquire locks related to both lists to guarantee linearizable access in
respect to both lists. The authors managed to use their language to express
synchronization constraints for the entire Java library, with acceptable
performance penalty incurred by the conservative nature of the type system. \\

figures out what subsets of data need to be generally atomic, or atomic in
respect to an operation, this information is encoded
exactly once, which is not the case when the programmer has to insert
synchronization code by hand. The policies are now known and enforced, even when
another programmer modifies code, they will not have to worry about correctly
synchronizing code (as long as no higher level synchonization is required). \\

% This whole thing needs to be mentioned earlier
Atomic sets also make the fact that consistency is a property of data explicit.
It is worth mentioning that other proposals support this claim. Many theoretical
models, such as [ref to coordination avoidance paper and try to locate the rest]
describe correctness invariants in terms of data states. A consistency level
guarantees invariant maintenance and, in that sense is also directly related to
data.

We take the idea of atomic sets, and generalize it to include several
consistency policies. This is done by annotating each set with the desired
consistency policy, and imposing restrictions on intersections between sets that
have different consistency policies. Our extensions will be explained in detail in Section 3.1.

\subsection{Cloud Types}
Cloud types (ref) are proposed to allow programmers write distributed applications by
declaring data items, and assigning them types that encode the following:
\begin{enumerate}
  \item Whether the data is local, or shared and replicated.
  \item The merge policy of concurrent updates.
\end{enumerate}
A language containing these types is also proposed, in order to help programmers
write simple mobile applications that require replication of state among mobile
devices and a cluster of servers. The design of the language guarantees eventual
consistency semantics, without needing much programmer intervention. The design
if the language makes solving common distributed data storage problems easier.
For instance, the language provide basic cloud types, along with higher level
operations that precisely define the behavior of the program when calling these
operations concurrently. From these types, compounds types, such as entities,
lists and sets can be defined, again with clearn and intuitive consistency
semantics. Furthermore, the design of arrays (which are more like maps in
regular programming langauges) in the language makes assigning
unique keys to entities, and deleting entities less problematic in the presence
of concurrent operations (mention somewhere that these are designed after Chen's
relations). We will now explain cloud types, entities and arrays in detail through
the example in Fig.smth

Other than guaranteeing eventual consistency,the language also provide strong consistency
when needed. Using cloud types, programmers can declare data to be replicated
and shared on the cloud, the semantics of eventual consistency and conflict
resolution are hence tied to the data itself. Fig.(...) shows an example of a
program written using cloud types. Cloud types are included in a formally
described, Javascript-like language.

(From here on, extreme spitting what is inside my mind)
In this language, data types are of two main categories, regular and cloud
types. Regular types can be used to pass parameters to functions, and as
indices. Their values are local and not replicated. Cloud types, on the other
hand Cannot be used as indices, escape manipulating functions or appear in
expressionss. They can be though of as terminals, where final results of
computations are written to or initially read from, and nothing else can be done
to them. Cloud types come with well-defined behavior for merging updates
proposed by concurrent modifications, in terms of fork-join automata. Compound
cloud types, which are array and entities (inspired by relational Java) can be
expressed in terms of more primitive cloud types (e.g. CInt, CString...etc).
Once the operations

talk about deletes, how array semantics solve certain problems...etc

What is nice about cloud types is that operations are naturally separated into
the regular generation and manipulation operations, which are local, and to
operations that actually commit data and make it available for sharing and
further manipulation, the language also restricts the concept of deletion, in
order to avoid 


\begin{figure}[tp]
\begin{lstlisting}
entity Customer {
   name: CString
}

array Product[id: string] {
  name: CString
  price: CInt
}

entity Order(customer: Customer) {
  time: CTime
  totalPrice: CInt
}

array CartItem[customer: Customer, product: Product] {
  quantity: CInt
}

array OrderItem[order: Order, product: Product] {
  quantity: CInt
  price: CInt
}

function addToCart(c: Customer, p: Product, q: int) {
  cartItem[c, p].quantity.add(q)
}

function deleteCustomer(c: Customer) {
  delete c
}

function submitOrder(customer: Customer) {
  var order = new Order(customer)
  order.time.set(now());

  foreach cartitem in entries CartItem.quantity 
    where cartitem.customer == customer {
      var oitem = OrderItem[order, cartitem.product]
      oitem.quantity = cartitem.quantity
      oitem.price = cartitem.quantity * cartitem.product.price
      cartitem.quantity.add(-oitem.quantity)
      order.totalprice.add(oitem.price)
  }
}

function showOrders(customer: Customer) {
  foreach order in all Order
    where order.Customer == customer
    orderby order.time {    
    Print("Order of" + order.time)
    
    foreach(i in all OrderItem) where i.order == order {
       Print(i.quantity + " " 
            + i.product.name + "for" + i.price)
    }
  }
}
\end{lstlisting}
\caption{Counter example using atomic sets}
\label{lst:ctStore}
\end{figure}


\begin{figure}[tp]
\begin{lstlisting}
array Seat [row: int, letter: string] {
  assignedTo: CString
}

function naiveReserve(seat: Seat, customer: string) {
  if(seat.assignedTo.get() == "")
    seat.assignedTo.set(customer)
  else 
    print("Reservation failed!")
}

function goodReserve( seat: Seat, customer: string) {
  seat.assignedTo.setIfEmoty(customer)
  flush
  if(seat.assignedTo.get() != customer) print("Reservation Failed!")
}
\end{lstlisting}
\caption{Cloud types compare-and-set for strong consistency}
\label{lst:acCmpSet}
\end{figure}

\begin{figure}[tp]
\begin{lstlisting}
class Counter {
  atomicset a;
  atomic(a) int val;
  int get() { return val; }
  void dec() {val--; }
  void inc { val++; }
}

Counter c = new Counter();
c.inc();
c.dec();
...
\end{lstlisting}
\caption{Counter example using atomic sets}
\label{lst:acCounter}
\end{figure}



\begin{figure}[tp]
\begin{lstlisting}
class Counter {
  (a) int val;
  int get() { synchronized { return val; } }
  void dec() { synchronized { val--; } }
  void inc { synchronized { val++; } }
}

Counter c = new Counter();
c.inc();
c.dec();
...
\end{lstlisting}
\caption{Counter example after compilation}
\label{lst:acCounterCompiled}
\end{figure}


\begin{figure}[tp]
\begin{lstlisting}
class List {
  atomicset a;
  atomic(a) int size;
  atomic(a) Entry header |b = this.a|
  ....
}
\end{lstlisting}
\caption{Linked List example with addAll method}
\label{lst:acList}
\end{figure}




































